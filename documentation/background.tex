\documentclass[]{article}
\usepackage[margin=.7in, portrait]{geometry}
%\usepackage{fullpage}
\usepackage{amsmath}
\usepackage{amssymb}
\usepackage{amsthm}
\usepackage{mathrsfs}
\usepackage[english]{babel}
\usepackage{multirow}
\usepackage{graphicx}
\newtheorem{stel}{Stelling}
\usepackage{float}
\usepackage{textcomp}
\usepackage{footmisc}
\usepackage[colorlinks=true, urlcolor=blue, linkcolor=blue, citecolor=blue, pdfborder={0 0 0}, linktoc=page]{hyperref}
\setlength{\parindent}{0 pt}
\usepackage{etoolbox}
\usepackage{scrextend}

\newcommand{\ba}{\begin{addmargin}[9mm]{0cm}}
\newcommand{\ea}{\end{addmargin}}
\newcommand{\cm}{\color{blue} \hspace{3mm}}


\title{\vspace{0cm} \huge 3D-RT: 3D Radiative Transfer \\
                          { \Large \vskip3mm Background}}

\author{\large Frederik De Ceuster}
\date{}


\begin{document}

\maketitle

\vskip6cm

\begin{abstract}
This report gives an overview of some technical aspects of the 3D-RT code. The goal is to motivate why and explain how some things are coded the way they are.
\end{abstract}

\vskip6cm

\tableofcontents

\newpage


\section{General}

3D-RT is a multi purpose 3D radiative transfer code

The code is mainly written in C with hsome features of C++.

\subsection{Storing multi-dimensional arrays as lists}

All multi-dimensional arrays in the code are stored as one-dimensional lists. On the lowest level, this is the case in every computer code. However, we chose to explicitly write the one-dimensional lists and define the relations between list index and the rows and columns.

\section{Ray tracing}

To be able to simulate 3-dimensional clouds of gas on a computer we must find a way to represent this system in computer language. The first step is to chop up space into a bunch of cells. The (continuous) physical variables like e.g. temperature and density are assumed to be constant within each cell.

\bigskip

Space is discretized into \texttt{ngrid} grid points.

\subsection{Efficiently storing the evaluation points}

THE ORIGIN IS NOT AN EVALUATION POINT!

\texttt{raytot[RINDEX(n,r)]} gives the total number of evaluation points on a ray \texttt{r} through a gridpoint \texttt{n}. Here the origin is not counted as an evaluation point. Otherwise we would store the origin each time as an evaluation point, resulting in \texttt{ngrid} times \texttt{NRAYS} unnecessary doubles. In the radiative transfer part of the code we do want to consider the origin as an eveluation point. Therefore we will systematically add one to \texttt{raytot} in that part of the code.

\subsection{Equivalent rays}
of

\section{Radiative transfer}




\section{Thermal balance}

\subsection{Heating}

\subsection{Cooling}

\subsubsection{Radiative cooling}








\end{document}
