\documentclass[]{article}
\usepackage[margin=1.1in, portrait]{geometry}
\usepackage{amsmath}
\usepackage{amssymb}
\usepackage{amsthm}
\usepackage{mathrsfs}
\usepackage[english]{babel}
\usepackage{multirow}
\usepackage{graphicx}
\newtheorem{stel}{Stelling}
\usepackage{float}
\usepackage{textcomp}
\usepackage{footmisc}
\usepackage[colorlinks=true, urlcolor=blue, linkcolor=blue, citecolor=blue, pdfborder={0 0 0}, linktoc=page]{hyperref}
\setlength{\parindent}{0 pt}
\usepackage{etoolbox}
\usepackage{scrextend}

\newcommand{\ba}{\begin{addmargin}[9mm]{0cm}}
\newcommand{\ea}{\end{addmargin}}
\newcommand{\cm}{\color{blue} \hspace{3mm}}

\newcommand{\kb}{k_{\text{B}}}
\newcommand{\D}{\text{d}}


\title{ The modern problem of computational radiative transfer }

\author{ Frederik De Ceuster }
\date{}


\begin{document}

\maketitle

\begin{abstract}
The aim is to formulate the modern problem of computational radiative transfer and explain how it is solved in the \texttt{Magritte} code.
\end{abstract}


\section{Problem definition}

Consider a multidimensional space with a scalar field $I(\textbf{x},\hat{\textbf{n}})$, depending on both the position $\textbf{x}$ and the viewing direction $\hat{\textbf{n}}$ in that space. Consider a

\subsection{Method of rays}

We choose to solve the problem by directly integrating the transfer equation along each ray. The result of this calculation is the intensity at a certain point in a certain direction.


\section{Input data}

Since we want to develop a general-purpose code, we need to make sure that it can handle many different types of input. We input can come from the output of a hydrodynamics code For every differemt type we will consider the best way to handle the

\subsection{Model input}


\subsection{Hydro output as input}

\subsubsection{AMR grid}

The simplest way to handle an AMR grid input is to use the centers of the grid cells as set input grid points $G$.

\subsubsection{SPH data}


\bibliography{library}
\bibliographystyle{ieeetr}



\end{document}
