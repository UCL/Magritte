\maketitle
% \makedeclaration

\begin{abstract} % 300 word limit
	Radiative transfer plays a key role in the dynamics, the chemistry and the energy balance of all kinds of astrophysical objects. It provides the radiative pressure to drive stellar winds, it affects the chemistry through various photoionization and photodissociation reactions, and it can very efficiently heat or cool very specific regions in the wind. Therefore it is essential in modelling these objects to properly account for all radiative processes and their interdependence. This however can be complicated by i) an intricate 3D geometrical structure shielding or exposing specific regions to radiation, ii) the scattering by dust or free electrons, and iii) the mixing in frequency space due to Doppler shifts caused by velocity gradients in the medium. The tight coupling between radiative transfer and the often very specialized and diverse dynamical and chemical models furthermore requires a modular code that can easily be integrated. To address these needs we will present \textsc{Magritte}: a new multidimensional accelerated general-purpose radiative transfer code. \textsc{Magritte} is especially designed to have a well scaling performance on various computer architectures.  Appended with a dedicated chemistry and thermal balance module it can self-consitently calculate the temperature field, chemical abundances and level populations. Magritte is a deterministic ray-tracing code that obtains the radiation field by solving the transfer equation along a fixed set of rays originating from each grid cell. It iteratively accounts for scattering and treats the full frequency space. This allows us to self-consistently model the chemistry and energy balance in stellar winds and perform synthetic observations. We will apply {\sc Margritte} by post-processing snapshots of hydrodynamical wind and bowshock simulations and will present its integration in self-consistent hydro-chemical AGB wind models.
\end{abstract}

% \begin{acknowledgements}
% I would like to thank the Institute for Astronomy in the Department of Physics and Astronomy at KU Leuven for their hospitallity. I gratefully acknowledge the financial support of the EPSRC iCASE studentship programme, Intel Corporation and Cray Inc.
% \end{acknowledgements}

\setcounter{tocdepth}{2}
% Setting this higher means you get contents entries for
%  more minor section headers.

\tableofcontents
% \listoffigures
% \listoftables
